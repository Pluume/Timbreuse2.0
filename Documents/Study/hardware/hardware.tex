\documentclass[10pt,a4paper,onecolumn]{article}
\usepackage[top=30pt,bottom=30pt,left=48pt,right=46pt]{geometry}
\usepackage{a4wide}
\usepackage{titling}
\setlength{\droptitle}{-5em}
\usepackage[version=3]{mhchem}
\usepackage{siunitx}
\usepackage{graphicx}
\usepackage{amsmath}
\usepackage[utf8]{inputenc}
\usepackage[english,francais]{babel}
\usepackage[backend=bibtex]{biblatex}
\usepackage[pdftex]{hyperref}
\setlength\parindent{0pt}
\usepackage{xcolor}
\hypersetup{colorlinks,urlcolor=blue}
\usepackage{fancyhdr}
\usepackage{lastpage}
\pagestyle{fancy}
\fancyhf{}
\chead{Francis Le Roy}
\lfoot{P1699/MCT}
\cfoot{\today}
\rfoot{Page \thepage \hspace{1pt} of \pageref{LastPage}}
\bibliography{bib}
\renewcommand{\labelenumi}{\alph{enumi}.}


\title{Rapport de test \\ Compatibilité du matériel \\ avec le logiciel de la Timbreuse}

\author{Francis \textsc{Le Roy}}

\date{\today}

\begin{document}

\maketitle
\thispagestyle{fancy}

\begin{center}
\begin{tabular}{l r}
Date de réalisation : & 4 Avril, 2017 \\
Projet : & P1699 \\
\end{tabular}
\end{center}


\section{Introduction}
La Timbreuse à été dévelloppé à l'aide du framework Node JS. Celui ci permet de développer des applications bureau en utilisant du Javascript.
Node JS va compiler puis interpreter le code : cette technique s'appelle le "Compile Just In Time". La premiere chose à vérifirer est la
compatibilité du logiciel avec Node JS.
\section{Installation}
\subsection{Systéme d'exploitation}
L'installation du système d'exploitation du Raspberry Pi s'effectue en inscrivant sur une puce Micro SD l'image télécharger deèuis le site officiel
de Raspberry Pi.
\subsection{Connexion à Internet}
Pour ce connecter à internet à travers le réseau du CPNV deux méthodes s'offrent à nous. La prémière consiste à demandé un la création d'un mots de passe
jetable à notre professeur référent puis à déverouiller l'IP que l'on souhaite à travers notre panel depuis l'intranet du CPNV. Ou bien de nous connecter
avec le script \texttt{internet.sh} fournit par le service informatique du CPNV. La dernière solution fonctionne plus souvent.

\subsection{Installation des paquets de base}
Il faut tout d'abord installer les dernières mise-à-jour du système pour nous protéger des dernière failles de sécurités découvertes.
Pour ce faire il faut taper la commande \texttt{sudo apt\- get update \& \&  sudo apt-get upgrade \& \&  sudo apt-get dist-upgrade}. Après l'execution de cette commande
il est recommandé de redémarrer le pi.

\subsection{Installation de Node JS}
Pour installer Node JS il faut tout d'abord executer cette commande : \texttt{curl -sL https://deb.nodesource.com/setup\_7.x | sudo -E bash -} puis celle-ci
\texttt{sudo apt install nodejs}. Si tout c'est bien passé, vous devriez pouvoir voir la version de Node JS installé depuis un terminal en tapant la commande \texttt{node -v}.

\subsection{Installation de électron}
Pour insaller electron, l'un des framework sur lequel la Timbreuse à été développé, il faut taper la commande \texttt{sudo npm install -g electron}.
Si tout ces biens passé, la commande \texttt{electron} doit afficher une fenêtre de test.

\section{Conclusion}
Le Raspberry Pi avec comme système d'exploitation Rapsbian
est compatabible avec le logiciel de la Timbreuse. Pour plus d'information, voir le guide d'installation de Node JS sur Raspberry pi de Dave Johnson \cite{thisdavej}

\printbibliography

\end{document}
