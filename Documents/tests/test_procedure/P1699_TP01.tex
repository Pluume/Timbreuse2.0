\documentclass[10pt,a4paper,onecolumn]{article}
\usepackage[top=30pt,bottom=30pt,left=48pt,right=46pt]{geometry}
\usepackage{a4wide}
\usepackage{titling}
\setlength{\droptitle}{-5em}
\usepackage[version=3]{mhchem}
\usepackage{siunitx}
\usepackage{graphicx}
\usepackage{amsmath}
\usepackage[utf8]{inputenc}
\usepackage[english,francais]{babel}
\usepackage[backend=bibtex]{biblatex}
\usepackage[pdftex]{hyperref}
\setlength\parindent{0pt}
\usepackage{xcolor}
\hypersetup{colorlinks,urlcolor=blue}
\usepackage{fancyhdr}
\usepackage{lastpage}
\pagestyle{fancy}
\fancyhf{}
\chead{Francis Le Roy}
\lfoot{P1699/MCT}
\cfoot{\today}
\rfoot{Page \thepage \hspace{1pt} of \pageref{LastPage}}
% \bibliography{bib}
\renewcommand{\labelenumi}{\alph{enumi}.}



\title{Procédure de test P1699\_TP01}

\author{Francis \textsc{Le Roy}}

\date{\today}

\begin{document}

\maketitle
\thispagestyle{fancy}

\begin{center}
\begin{tabular}{l r}
Date de réalisation : & 13 Mai, 2017 \\
Projet : & P1699
\end{tabular}
\end{center}

\section{Introduction}
Cette procédure consiste à faciliter la lecture d'autres rapport de testes afin de modulariser au maximum la documentation et la rendre la plus clair possible.
\section{Organisation visuelle}
Les méthodes de test sont similaires entre tout les tests concernant les panneaux des administrateurs, des enseignants et des élèves. Sur ces panneaux, une interface parentes que l'on nomera «Conteneur» contiens des liens vers les différentes page du panel correspondant. 
Chaque lien redirige la page enfant que l'on nomera «Hôte».\\

\section{Méthode de tests}
Pour pouvoir tester les élèments de l'interface graphique il est nécessaire de se connecter sur le logiciel de la timbreuse avec un identifiant et un mot de passe valide pour pouvoir accéder au panneaux correspondant au rang de l'utilisateur connecté. Un fois sur son panneau, de nombreux controle sous la forme de bouton, de calendrier, de boite de texte ou de boite à cocher son disponible. \\
Pour tester la fonctionnalité étudiée, l'utilisateur doit naviguer dans le logiciel de la timbreuse jusqu'a pouvoir accéder au contrôle qui va activé la fonction testé.

\section{Interpretation des résultats obtenus}
\subsection{Succès}
On identifie un succès généralement par un affichage de informations recherchés lors de l'execution de la fonction étudié ou bien par un boite de texte de fond vert au sommet de l'application avec un message décrivant le résultat de la fonction. Cette boîte de texte disparaît au bout de 2 secondes.
\subsection{Erreurs}
On identifie généralement une erreur par une absence de réaction après avoir déclenché la fonction. Si la fonction cette executé correctement mais qu'une erreur connu à été détecté, alors une boite de texte de fond rouge apparaîtra pendant 15 secondes avec un cour message décrivant l'erreur.
\section{Conclusion}
Si la fonctionnalité à été un succès, on peut donc définir cette fonctionnalité comme opérationnelle. 
% \printbibliography

\end{document}

