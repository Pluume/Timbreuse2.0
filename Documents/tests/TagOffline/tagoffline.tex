\documentclass[10pt,a4paper,onecolumn]{article}
\usepackage[top=30pt,bottom=30pt,left=48pt,right=46pt]{geometry}
\usepackage{a4wide}
\usepackage{titling}
\setlength{\droptitle}{-5em}
\usepackage[version=3]{mhchem}
\usepackage{siunitx}
\usepackage{graphicx}
\usepackage{amsmath}
\usepackage[utf8]{inputenc}
\usepackage[english,francais]{babel}
\usepackage[backend=bibtex]{biblatex}
\usepackage[pdftex]{hyperref}
\setlength\parindent{0pt}
\usepackage{xcolor}
\hypersetup{colorlinks,urlcolor=blue}
\usepackage{fancyhdr}
\usepackage{lastpage}
\pagestyle{fancy}
\fancyhf{}
\chead{Francis Le Roy}
\lfoot{P1699/MCT}
\cfoot{\today}
\rfoot{Page \thepage \hspace{1pt} of \pageref{LastPage}}
% \bibliography{bib}
\renewcommand{\labelenumi}{\alph{enumi}.}



\title{Rapport de test \\ Timbrage hors connexion}

\author{Francis \textsc{Le Roy}}

\date{\today}

\begin{document}

\maketitle
\thispagestyle{fancy}

\begin{center}
\begin{tabular}{l r}
Date de réalisation : & 4 Avril, 2017 \\
Projet : & P1699 \\
\end{tabular}
\end{center}

\section{Introduction}
Il peut arriver que le réseau du CPNV tombe en panne pour une raison ou pour autre. Durant cette période de deconnexion les élèves doivent pouvoir timbrer pour répondre à ce problème, la timbreuse doit être capable d'effectué des actions de timbrage différé au moment de la reconnexion.
\section{Procédure de test}
Tout d'abord, timbré sur une timbreuse à l'aide d'une carte d'un élève que l'on gardera durant toute l'expérience. Gardez en mémoire l'état actuelle de l'élève que vous venez de faire timbrer. Éteignez le serveur et effectué 3 timbrage avec la carte de l'élève sur la timbreuse. Rebranchez le serveur et attendez quelques secondes que les timbreuses se reconnecte. Lorsque le message affichant que la timbreuse est déconnecté s'en va, timbrez de nouveau. Si l'état est le même que lorsque vous aviez timbré en premier, alors le test est un succès.
\section{Résultat obtenue}
En réalité après la reconnexion le serveur va traiter chacune des requêtes une par une. En revanche il ne renvoie pas les informations relatives à l'élève à la timbreuse. Si plusieurs personne ont timbré, ca n'aurait pas de sens de renvoyer les informations de la dernière personne traité par le système.
\section{Conclusion}
Cette fonctionnalité est donc opérationnelle.\\
{\color{red}Attention ! Les élèves ne doivent pas timbrer sur des timbreuses différentes. Cela pourrait provoquer des résultats inattendu.}
% \printbibliography

\end{document}
