\documentclass[10pt,a4paper,onecolumn]{article}
\usepackage[top=30pt,bottom=30pt,left=48pt,right=46pt]{geometry}
\usepackage{a4wide}
\usepackage{titling}
\setlength{\droptitle}{-5em}
\usepackage[version=3]{mhchem}
\usepackage{siunitx}
\usepackage{graphicx}
\usepackage{amsmath}
\usepackage[utf8]{inputenc}
\usepackage[english,francais]{babel}
\usepackage[backend=bibtex]{biblatex}
\usepackage[pdftex]{hyperref}
\setlength\parindent{0pt}
\usepackage{xcolor}
\hypersetup{colorlinks,urlcolor=blue}
\usepackage{fancyhdr}
\usepackage{lastpage}
\pagestyle{fancy}
\fancyhf{}
\chead{Francis Le Roy}
\lfoot{P1699/MCT}
\cfoot{\today}
\rfoot{Page \thepage \hspace{1pt} of \pageref{LastPage}}
% \bibliography{bib}
\renewcommand{\labelenumi}{\alph{enumi}.}



\title{Rapport de test \\ Exportation des CSV \\ lorsque les élèves timbres}

\author{Francis \textsc{Le Roy}}

\date{\today}

\begin{document}

\maketitle
\thispagestyle{fancy}

\begin{center}
\begin{tabular}{l r}
Date de réalisation : & 2 Mars, 2017 \\
Projet : & P1699 \\
\end{tabular}
\end{center}

\section{Introduction}
Lorsque les professeurs veulent avoir accès au CSV, il doivents passer un tag maître à une timbreuse esclave, celle-ci va l'envoyer au serveur. Le serveur répondra en donnant l'ordre à la timbreuse esclave de copier tous ses CSV sur les supports de stockage amovibles disponibles. Ce test ne vérifie que la fonction exportant les données sur les supports externes.
\section{Résultats attendus}
Un dossier daté devrait être crée sur la clé USB. De plus, tous les CSV de la timbreuse devraient être stockés sur la clé dans le dossier daté.
\section{Procédure de test}
Il faut utiliser le fichier \href{run:../../../test/csv_export.js}{script} Node JS, qui va effectuer la fonction exportant les CSV. Si la fonction s'execute correctement, le message "Done" sera affiché dans la console
\section{Résultats obtenus}
Lorsque le script est executé, tous les CSV sont copiés dans un dossier daté sur la clé USB. Le message "Done" est apparu dans la console.
\section{Conclusion}
Cette fonctionnalité est donc opérationnelle.
% \printbibliography

\end{document}
